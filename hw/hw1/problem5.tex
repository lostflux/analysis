\begin{problem}
  Give an example of a countably infinite collection of infinite sets
  $A_1, A_2, A_3, ...$ with $A_j \cap A_k$ being infinite for all
  $j$ and $k$ such that $\displaystyle \bigcap_{j=1}^\infty A_j$ is nonempty
  and finite.
\end{problem}
\begin{answer}
  Let $A = \Z_{\geq 0} = \set{0, 1, 2, 3, \ldots}$ be the set of all nonnegative integers.
  For each $i \in \N$, define $A_i = \set{a \in A \mid i \text{ divides } a}$.
  That is, the $A_i$ is the set of all nonnegative integers that are divisible by $i$. 
  Then:
  \begin{align*}
    A_1 &= \set{0, 1, 2, 3, \ldots} = A \\
    A_2 &= \set{0, 2, 4, 6, \ldots} \\
    A_3 &= \set{0, 3, 6, 9, \ldots} \\
    &\vdots
  \end{align*}
  \begin{enumroman}
    \item First, we'll show that $A_j \cap A_k$ is infinite for all $j$ and $k$.
      For any arbitrary $j, k \in \N$;
      \[
        (A_j \cap A_k) = \set{0, jk , 2 \cdot jk, 3 \cdot jk, \ldots} = \set{ n \cdot jk \mid n \in \Z_{\geq 0}}.
      \]
      Since $\Z_{\geq 0}$ is infinite, $(A_j \cap A_k)$ is also infinite. \\
    \item We'll then show that $\displaystyle \bigcap_{j=1}^\infty A_j$ is nonempty yet finite.
      \begin{enumarabic}
        \item $\displaystyle \bigcap_{j=1}^\infty A_j$ is nonempty.
          Since $0 \in A_j$ for all $j \in \N$, then $0 \in \displaystyle \bigcap_{j=1}^\infty A_j$. \\
        \item $\displaystyle \bigcap_{j=1}^\infty A_j$ is finite.
          Precisely, $\displaystyle \bigcap_{j=1}^\infty A_j = \set{0}$.
          Suppose, for contradiction, that $\displaystyle \bigcap_{j=1}^\infty A_j$
          contains an element $x \neq 0$.
          This implies that $x \in A_j$ for all $j \in \N$,
          thus $x$ is divisible by every positive integer $j$.
          Take $y = 2x$, then $\displaystyle \frac{x}{y} = \frac{x}{2x} = \frac{1}{2}$,
          so $x$ is not divisible by $y$, contradicting the assumption that $x$
          is divisible by every positive integer.
      \end{enumarabic}
    \item Finally, we show that the collection of sets $A_1, A_2, \ldots$ is countably
      infinite by constructing a correspondence (bijection) with $\N$.
      Let $\frakA = \set{A_1, A_2, \ldots}$.
      Then, define the function $\psi : \N \to \frakA$ as follows:
      \begin{align*}
        \psi : \N &\to \frakA \\
        1 &\mapsto A_1 \\
        2 &\mapsto A_2 \\
        3 &\mapsto A_3 \\
        &\vdots
      \end{align*}
  \end{enumroman}
\end{answer}
