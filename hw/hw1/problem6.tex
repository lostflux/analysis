\begin{problem}
  Prove that a function is invertible if and only if it is a bijection.
\end{problem}

\begin{answer}
  \begin{definition}
    A function $f : A \to B$ is \emph{invertible} if there exists a function
    $g : B \to A$ such that $g \circ f = \id_A$ and $f \circ g = \id_B$.
  \end{definition}

  \step
  For the sake of contradiction, suppose that $f : A \to B$ is invertible
  but not a bijection. Let $g : B \to A$ be its inverse.
   \begin{enumroman}
    \item Suppose $f$ is not an injection.
      Then there exists some $a_1, a_2 \in A$,
      $a_1 \neq a_2$, and some $b \in B$ such that $f(a_1) = f(a_2) = b$.
      Since $g$ is the inverse of $f$, $g(f(x)) = x$ for all $x \in A$,
      which implies that $g(b) = a_1$ and $g(b) = a_2$.
      However, a function cannot map an element to two different elements,
      so either $f$ is an injection or it must not be invertible.

    \item Suppose $f$ is not a surjection.
      Then there exists some $b \in B$ such that 
      $f(a) \neq b$ for all $a \in A$.
      However, $g$, the inverse of $f$, acts from $B$ to $A$,
      so it must assign some element $a' \in A$ to $b$.
      This implies that $g(b) = a'$ for some $a' \in A$.
      But $g$ was taken to be the inverse of $f$, so it must map
      $f(a') \mapsto a'$, where $f(a') \neq b$.
      Therefore $g$ is not an injection (and, by the previous result,
      not invertible). This contradicts the assumption that $g = f^{-1}$
      as that would imply that $g$ is also invertible (with $f$ being its inverse).
    
   \end{enumroman}


\end{answer}
