\documentclass[11pt]{amsart}

% Include the macros file from `../common'
\input{~/latex-common/macros.tex}

\pagestyle{fancy}                       % fancy (allow headers, footers)
\fancyhf{}                              % clear all header/footer settings.
\cfoot{\thepage}                        % set page-numbers in footer.
\renewcommand{\headrulewidth}{0pt}
% \lhead{\textit{\textbf{ Amittai, S}}}   % set name in header, left.
% \rhead{\textsc{Math 63}}         % set class name in header, right.

\renewcommand{\theenumi}{\alph{enumi}}

\begin{document}
\setlength{\headheight}{13.0pt}
\setlength{\footskip}{13.0pt}


% TITLE
\newdate{due-date}{21}{02}{2024}
\PSET{7 --- \displaydate{due-date}}
  {Winter 2024}
  {Erchenko}
  {Amittai Siavava}
  {Math 63: Real Analysis}

% CREDIT STATEMENT
\CreditStatement{
  I worked on these problems alone,
  with reference to class notes and the following books:
  \begin{enumerate}
    \item \textbf{\textit{Introduction to Analysis}} by Maxwell Rosenlicht
  \end{enumerate}
}

\bigskip

\noindent 
\begin{center} {\bf Problems} \end{center}

\begin{enumarabic}
  \item Why did you choose to take Math 63?
  
    \begin{Answer}
      I am interested in learning more about the foundations of mathematics
      and how a strong grounding in analysis would be very useful in that regard.
    \end{Answer}
  
  \item What career looks interesting to you after getting your bachelor's degree?
  
    \begin{Answer}
      I am a double major in mathematics and computer science at Dartmouth.
      I am most likely going to pursue a career in software engineering
      or quantitative trading since I have had great experiences with both.
    \end{Answer}
  
  \item What prior mathematics classes did you take?
  
    \begin{Answer}
      \begin{enumroman}
        \item MATH-11: Multivariable Calculus
        \item MATH-22: Linear Algebra
        \item MATH-23: Differential Equations
        \item MATH-69: Logic
        \item MATH-71: Abstract Algebra
        \item MATH-75: Cryptography
        \item MATH-100: Game Theory
      \end{enumroman}

      I have also done some coursework in theoretical computer science,
      including Discrete Mathematics (COSC-30), Algorithms (COSC-31),
      and Theory of Computation (COSC-39)
    \end{Answer}
  
  \newpage
  \item What methods of proof do you feel comfortable with?
  
    \begin{Answer}
      Direct proof, induction, contradiction, contrapositive.
    \end{Answer}

  \item What is your favorite math fact? Why?
  
    \begin{Answer}
      I do not know if I have a specific favorite math fact.
      But I love that there's a direct correspondence between
      decidability in logic and computability in computer science
      -- they are almost different expressions of the same thing.
    \end{Answer}
  
  \item Did you use \TeX\ before?
    If so, how was your experience?

    \begin{Answer}
      I have used \TeX\ before in various mathematics and computer science courses
      at Dartmouth. I also have used it for my resume
       and most recently on a few posts on my blog
       (such as \href{https://amittai.space/math/2023-02-riemann-zeta-properties}{this one}).
    \end{Answer}

  \item What helps you understand mathematics and get comfortable with new material?
    
    \begin{Answer}
      Writing a lot (proofs, equations, explanations, etc).
      I think a lot of things stick more when I write them down
      as I study them. In this sense, I appreciate that you write down a lot of
      theorems, lemmas, and proofs on the board as you go over them
      since it helps me follow along and understand the material better.
    \end{Answer}
  
  \item What do you expect from this course?
  
    \begin{Answer}
      I \emph{generally} hope develop a more nuanced understanding
      and perspective of mathematics.
      I think sometimes a fact or property in a given field
      gives insight into something else in a somewhat unrelated field.
      I'm mainly excited to learn and hopefully love
      the mathematics presented in this course,
      but I am also keeping an open mind on how the material connects with
      other topics and fields of interest.

      I am also taking complex analysis in the Spring and I
      wonder how the two courses will complement or differ from
      each other.
    \end{Answer}
\end{enumarabic}

\vfill

\end{document}
