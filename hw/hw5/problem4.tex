\begin{problem}
  Let $(E, d)$ and $(E', d')$ be metric spaces, $f : E \to E'$ be a function,
  and let $p \in E$. Define the oscillation of $f$ at $p$ to be
  \[
    \inf \set{
      a \in \R \given
      \text{
        $\exists$ open ball $B_r(p) \in E$
        such that $\forall x, y \in B_r(p)$,
        $d'(f(x), f(y)) \leq a$
      }
    } \]
    if the set is nonempty, and $+\infty$ otherwise.
    Prove that $f$ is continuous at $p$ if and only if the oscillation of $f$ at $p$ is $0$,
    and that for any real number $\epsilon$, the set of points of $E$
    at which the oscillation of $f$ is at least $\epsilon$ is closed.
\end{problem}
\begin{answer}
  \begin{claim}
    $f$ is continuous at $p$ if and only if the oscillation of $f$ at $p$ is $0$.
    \begin{proof}
      First, we shall show that if $f$ is continuous at $p$, then the oscillation of $f$ at $p$ is $0$.
      Then, we shall show that if the oscillation of $f$ at $p$ is $0$, then $f$ is continuous at $p$.
      \begin{enumarabic}
        \item ($\Rightarrow$)
          Suppose $f$ is continuous at $p$.
          Let $\epsilon > 0$.
          Then there exists $\delta > 0$ such that
          for all $x \in E$ with $d(x, p) < \delta$,
          $d'(f(x), f(p)) < \epsilon$.
          This means that for all $x, y \in B_\delta(p)$, $d'(f(x), f(y)) < \epsilon$.
          Knowing that $f$ is contunyous at $p$, we can pick $\epsilon$ such that
          the corresponding $\delta$ is arbitrarily small, so the oscillation of $f$ at $p$ is $0$.
        \item ($\Leftarrow$)
          Suppose the oscillation of $f$ at $p$ is $0$.
          This means that for all $\epsilon > 0$, there exists an open ball $B_r(p)$ in $E$ such that
          for all $x, y \in B_r(p)$, $d'(f(x), f(y)) \leq \epsilon$.
          Set $\delta = r$, and we have that for all $x \in E$ with $d(x, p) < \delta$,
          $d'(f(x), f(p)) < \epsilon$, which is the definition of continuity.
          Therefore, $f$ is continuous at $p$.
      \end{enumarabic}
    \end{proof}
  \end{claim}

  \begin{claim}
    For any real number $\epsilon$, the set of points of $E$
    at which the oscillation of $f$ is at least $\epsilon$ is closed.
    \begin{proof}
      Let $A = \set{p \in E \given \text{oscillation of $f$ at $p$} \geq \epsilon}$.
      We will show that $A$ is closed by showing that every limit point of $A$ is in $A$.
      Let $p \in E$ be a limit point of $A$ and suppose, for contradiction,
      that $p \notin A$.
      Then, the oscillation of $f$ at $p$ is less than $\epsilon$ (by the definition of $A$).
      This means that there exists an open ball $B_r(p)$ in $E$ such that
      for all $x, y \in B_r(p)$, $d'(f(x), f(y)) < \epsilon$.
      But since $p$ is a limit point of $A$, there exists $q \in A$ such that $q \in B_r(p)$.
      Then, the oscillation of $f$ at $q$ is at least $\epsilon$ (by the definition of $A$, since $q \in A$).
      This means that there exists an open ball $B_s(q)$ in $E$ such that
      for all $x, y \in B_s(q)$, $d'(f(x), f(y)) \geq \epsilon$.
      Pick a point $a \in B_s(q)$ such that $a \in B_r(p)$,
      then we have:
      \begin{enumarabic}
        \item $d(f(a), f(q)) \geq \epsilon$ (since $q \in A$).
        \item $d(f(a), f(q)) \geq \epsilon$ (since $a, q \in B_r(p)$).
      \end{enumarabic}
      Clearly, both of these statements cannot be true at the same time.
      This is a contradiction, so our assumption that $p \notin A$ must be false.
      Therefore, $p \in A$, so $A$ is closed.
    \end{proof}
  \end{claim}
\end{answer}
