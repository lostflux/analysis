\begin{problem}
  Let $f : [a, b] \to \R$ with $a < b$ be continuous.
  Show that functions
  \[
    m(x) = \inf \set{ f(y) \given a \leq y \leq x } \quad \text{and} \quad
    M(x) = \sup \set{ f(y) \given a \leq y \leq x }
  \]
  are continuous on $[a, b]$.
\end{problem}

\begin{answer}
  We will show that $m$ is continuous on $[a, b]$.
  The proof for $M$ is analogous.

  \begin{claim}
    The function $m$ is continuous on $[a, b]$.
    \begin{proof}
      Let $x_0 \in [a, b]$.
      We need to show that
      $\displaystyle \lim_{x \to x_0} m(x) = m(x_0)$.


      First, note that:
      \begin{enumarabic}
        \item $m(x) \leq f(x)$ for all $x \in [a, b]$, by definition of $m$,
          so $m$ is bounded above by $f$.
        \item $m(x)$ is monotonically decreasing on $[a, b]$.
          This also follows from the definition of $m$, since it is the
          infimum of all values of $f$ at points less than or equal to $x$.
      \end{enumarabic}
    
      Since $f$ is continuous on $[a, b]$, we have
      $\displaystyle \lim_{x \to x_0} f(x) = f(x_0)$.
      Thus, for any $x_0 \in [a, b]$ and every $\epsilon > 0$
      there exists $\delta > 0$ such that
      $\crim{\abs{x - x_0} < \delta} \implies \crim{\abs{f(x) - f(x_0)} < \epsilon}$.
      For any such $\epsilon$ and $\delta$, pick $x$ so that $\abs{x - x_0} < \delta$
      (and, consequently, $\abs{f(x) - f(x_0)} < \epsilon$).
      We claim that $\abs{m(x) - m(x_0)} < \epsilon$.

      \step
        Since $f$ is continuous,
        by definition of $m$ as an infimum of all previous values of $f$,
        if $x < x_0$ and $m(x) \neq m(x_0)$
        then $m(x) = f(y)$ for some $y \in [x, x_0]$. Since all points in that range
        are within $\delta$ of $x_0$, we have $\abs{f(y) - f(x_0)} < \epsilon$.
        Furthermore, since $m(x)$ is monotone decreasing,
        $m(x_0) \leq m(x)$ when $x < x_0$, so $m(x_0) = f(z)$ for some $z \in [x, x_0]$
        if $m(x) \neq m(x_0)$. Thus, either $m(x) = m(x_0)$ and $\abs{m(x) - m(x_0)} = 0$
        or 
        \[ 
          \abs{m(x) - m(x_0)} = \abs{f(y) - f(z)} < \epsilon \text{ for some } x_0 - \delta < x < y, z < x_0.
        \]
        
        Since this works for any $\epsilon$ and any $x_0$,
        we have $\abs{m(x) - m(x_0)} < \epsilon$.
    \end{proof}
  \end{claim}

  \newpage
  \begin{claim}
    The function $M$ is continuous on $[a, b]$.
    \begin{proof}
      The proof is analogous to the proof for $m$.
      Since $M(x) \geq f(x)$ for all $x \in [a, b]$, by definition of $M$,
      $M$ is bounded below by $f$.
      Also, $M(x)$ is monotonically increasing on $[a, b]$.
      This also follows from the definition of $M$, since it is the
      supremum of all values of $f$ at points less than or equal to $x$.
      Since $f$ is continuous on $[a, b]$, we have
      $\displaystyle \lim_{x \to x_0} f(x) = f(x_0)$.
      Thus, for any $x_0 \in [a, b]$ and every $\epsilon > 0$
      there exists $\delta > 0$ such that
      $\crim{\abs{x - x_0} < \delta} \implies \crim{\abs{f(x) - f(x_0)} < \epsilon}$.
      For any such $\epsilon$ and $\delta$, pick $x$ so that $\abs{x - x_0} < \delta$
      (and, consequently, $\abs{f(x) - f(x_0)} < \epsilon$).
      We claim that $\abs{M(x) - M(x_0)} < \epsilon$.

      \step
        Since $f$ is continuous,
        by definition of $M$ as an supremum of all previous values of $f$,
        if $x < x_0$ and $M(x) \neq M(x_0)$
        then $M(x) = f(y)$ for some $y \in [x, x_0]$. Since all points in that range
        are within $\delta$ of $x_0$, we have $\abs{f(y) - f(x_0)} < \epsilon$.
        Furthermore, since $M(x)$ is monotone increasing,
        $M(x_0) \geq M(x)$ when $x < x
        0$, so $M(x_0) = f(z)$ for some $z \in [x, x_0]$
        if $M(x) \neq M(x_0)$. Thus, either $M(x) = M(x_0)$ and $\abs{M(x) - M(x_0)} = 0$
        or
        \[ 
          \abs{M(x) - M(x_0)} = \abs{f(y) - f(z)} < \epsilon \text{ for some } x_0 - \delta < x < y, z < x_0.
        \]

        Since this works for any $\epsilon$ and any $x_0$,
        we have $\abs{M(x) - M(x_0)} < \epsilon$.
    \end{proof}

  \end{claim}
\end{answer}
