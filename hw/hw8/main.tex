% vim: set spell spelllang=en_us:
\documentclass[10pt, final]{article}
% \documentclass[11pt]{amsart}

\input{~/common}
% % ENCODING
\usepackage[utf8]{inputenc}
\usepackage{pmboxdraw}
% \usepackage{pmboxdraw-extras}

% text alignment
\usepackage{array, ragged2e}

% URLs
\usepackage{hyperref}

\usepackage{xargs}

% Sets
\newcommand{\FF}{\mathbb{F}}
\newcommand{\NN}{\mathbb{N}}
\newcommand{\QQ}{\mathbb{Q}}
\newcommand{\RR}{\mathbb{R}}
\newcommand{\ZZ}{\mathbb{Z}}

\newcommand{\F}{\mathbb{F}}
\newcommand{\N}{\mathbb{N}}
\newcommand{\Q}{\mathbb{Q}}
\newcommand{\R}{\mathbb{R}}
\newcommand{\Z}{\mathbb{Z}}
\newcommand{\C}{\mathbb{C}}
\renewcommand{\b}{\{0,1\}}

% SET THEORY
\newcommand{\nequiv}{\not\equiv}
\renewcommand{\notin}{\not\in}

% FUNCTIONS
\usepackage{amsmath}
\DeclareMathOperator{\cost}{cost}
\DeclareMathOperator{\len}{len}
\DeclareMathOperator{\rank}{rank}
\DeclareMathOperator{\sgn}{sgn}
\DeclareMathOperator{\wt}{wt}

% ALGEBRA
\newcommand{\GL}{\mathrm{GL}}
\DeclareMathOperator{\id}{id}
\DeclareMathOperator{\M}{M}
\DeclareMathOperator{\SL}{SL}
\DeclareMathOperator{\Syl}{Syl}
\DeclareMathOperator{\tr}{tr}
\DeclareMathOperator{\ev}{ev}

% Combinatorics

% Probability
\newcommand{\Hd}{\texttt{\color{BrickRed}H}}
\newcommand{\Tl}{\texttt{T}}
\newcommand{\EE}{\mathop{\mathbb{E}}}
\newcommand{\PP}{\mathop{\mathbb{P}}}
\newcommand{\indic}{\mathbb{1}}
\DeclareMathOperator{\Var}{Var}

% End of proof marker
\newcommand{\qedblack}{\hfill\ensuremath{\blacksquare}}
\newcommand{\qedwhite}{\hfill\ensuremath{\square}}

% Calligraphic caps
\newcommand{\calA}{\mathcal{A}}
\newcommand{\calB}{\mathcal{B}}
\newcommand{\calC}{\mathcal{C}}
\newcommand{\calD}{\mathcal{D}}
\newcommand{\calE}{\mathcal{E}}
\newcommand{\calF}{\mathcal{F}}
\newcommand{\calI}{\mathcal{I}}
\newcommand{\calL}{\mathcal{L}}
\newcommand{\calO}{\mathcal{O}}
\newcommand{\calP}{\mathcal{P}}
\newcommand{\calS}{\mathcal{S}}
\newcommand{\calT}{\mathcal{T}}
\newcommand{\calU}{\mathcal{U}}
\newcommand{\calV}{\mathcal{V}}
\newcommand{\calX}{\mathcal{X}}
\newcommand{\calY}{\mathcal{Y}}
\newcommand{\calZ}{\mathcal{Z}}

% Fraktur letters
\newcommand{\frakA}{\mathfrak{A}}
\newcommand{\frakB}{\mathfrak{B}}
\newcommand{\frakC}{\mathfrak{C}}
\newcommand{\frakD}{\mathfrak{D}}
\newcommand{\frakE}{\mathfrak{E}}
\newcommand{\frakF}{\mathfrak{F}}
\newcommand{\frakG}{\mathfrak{G}}
\newcommand{\frakH}{\mathfrak{H}}
\newcommand{\frakI}{\mathfrak{I}}
\newcommand{\frakJ}{\mathfrak{J}}
\newcommand{\frakK}{\mathfrak{K}}
\newcommand{\frakL}{\mathfrak{L}} 
\newcommand{\frakM}{\mathfrak{M}}
\newcommand{\frakN}{\mathfrak{N}}
\newcommand{\frakO}{\mathfrak{O}}
\newcommand{\frakP}{\mathfrak{P}}
\newcommand{\frakQ}{\mathfrak{Q}}
\newcommand{\frakR}{\mathfrak{R}}
\newcommand{\frakS}{\mathfrak{S}}
\newcommand{\frakT}{\mathfrak{T}}
\newcommand{\frakU}{\mathfrak{U}}
\newcommand{\frakV}{\mathfrak{V}}
\newcommand{\frakW}{\mathfrak{W}}
\newcommand{\frakX}{\mathfrak{X}}
\newcommand{\frakY}{\mathfrak{Y}}
\newcommand{\frakZ}{\mathfrak{Z}}

% Boldface letters
\newcommand{\ba}{\mathbf{a}}
\newcommand{\bb}{\mathbf{b}}
\newcommand{\bp}{\mathbf{p}}
\newcommand{\bq}{\mathbf{q}}
\newcommand{\br}{\mathbf{r}}
\newcommand{\bu}{\mathbf{u}}
\newcommand{\bv}{\mathbf{v}}
\newcommand{\bx}{\mathbf{x}}
\newcommand{\by}{\mathbf{y}}
\newcommand{\bz}{\mathbf{z}}
\newcommand{\tbp}{\mathbf{\widetilde{p}}}

% Special math symbols
\newcommand{\eps}{\varepsilon}
\newcommand{\ceq}{\subseteq}
\newcommand{\ang}[1]{\langle{} #1 \rangle}
\newcommand{\ceil}[1]{\lceil{} #1 \rceil}
\newcommand{\floor}[1]{\lfloor{} #1 \rfloor}

% Problem names and other small-caps constants
\newcommand{\inv}{\textsc{inv}\xspace}

% Useful for marking steps of a derivation to explain later
\newcommand{\circled}[1]{\raisebox{.5pt}{\textcircled{\raisebox{-.1pt}{\scriptsize #1}}}}


% Page size and margins
% \usepackage[left=1in,right=1in,top=1.3in,bottom=1.3in,nofoot]{geometry}
\usepackage{fancyhdr}   % for fancy header
\usepackage{fancyvrb}   % for fancy verbatim
\usepackage{graphicx}   % for including images
\usepackage{enumerate}  % for enumerating lists
\usepackage{enumitem}
\usepackage[rgb, dvipsnames]{xcolor}
\usepackage{tcolorbox}

\usepackage{multicol}

% Answer BOX
\usepackage{microtype}
\usepackage{mdframed}
\newmdenv[%
  skipabove=6pt,
  skipbelow=6pt,
  innertopmargin=6pt,
  leftmargin=-5pt,
  rightmargin=-5pt, 
  innerleftmargin=5pt,
  innerrightmargin=5pt,
  backgroundcolor=black!10
]{Answer}%


% Header BOX
\newcommand{\handout}[6]{
  \noindent
  \begin{center}
  \setlength{\fboxrule}{1.2pt}
  \framebox{
    \vbox{
      \hbox to 5.78in { \textbf{#6} \hfill {\bf #2} }
      \vspace{4mm}
      \hbox to 5.78in { {\Large \hfill {\textbf{ #5 }}  \hfill} }
      \vspace{2mm}
      \hbox to 5.78in { {\textit{\textbf{#3 \hfill #4}}} }
    }
  }
  \setlength{\fboxrule}{0.2pt}
  \end{center}
  \vspace*{4mm}
}

% Header BOX
\newcommand{\PSET}[5]{\handout{#1}{#2}{Prof.\ #3}{Student: #4}{PSET #1}{#5}}

\newcommand{\homework}[5]{\handout{#1}{#2}{Prof.\ #3}{Student: #4}{Homework assigned #1}{#5}}

\newcommand{\exam}[5]{\handout{#1}{#2}{Prof.\ #3}{Student: #4}{Exam #1}{#5}}

\newcommand{\reading}[5]{\handout{#1}{#2}{Prof.\ #3}{Student: #4}{Reading assigned #1}{#5}}

% Credit Statement
\newcommand{\CreditStatement}[1]{
  \noindent
  \begin{center} {
    \bf Credit Statement
  }
  \end{center}
  { #1 }
}

% Problem Counter
\newenvironment{problem}[1][]%
{%
\ifx&#1&%
  % #1 is empty
  \stepcounter{problem}
\else
  % #1 is nonempty
  \setcounter{problem}{#1}
\fi
\setcounter{section}{\value{problem}}
\vspace{.2cm} \noindent {\center  \textbf{\\ Problem \arabic{problem}.\\}} {\noindent}~%
}{%
% \vspace{.2cm}%
}
% Package Imports
\usepackage{amssymb,amsthm,amsmath,amstext}
\usepackage{mathdots} % for \dots
  % \dotsc -- dots with commas.
  % \dotsb -- dots with binary operators.
  % \dotsm -- multiplication dots.
  % \dotsi -- dots with integrals.
  % \dotso -- "other dots".
\usepackage{wasysym, stackengine, makebox, tikz-cd}
\newcommand\isom{\mathrel{\stackon[-0.1ex]{\makebox*{\scalebox{1.08}{\AC}}{=\hfill\llap{=}}}{{\AC}}}}
\newcommand\nvisom{\rotatebox[origin=cc] {-90}{$ \isom $}}
\newcommand\visom{\rotatebox[origin=cc] {90} {$ \isom $}}


% Custom colors
\definecolor{crimson}{rgb}{0.86, 0.08, 0.24}
\definecolor{teal}{rgb}{0.0, 0.5, 0.5}
\definecolor{zaffre}{rgb}{0.0, 0.08, 0.66}
\definecolor{DarkOliveGreen}{rgb}{0.33, 0.42, 0.18}
\newcommand{\crim}{\textcolor{crimson}}
\newcommand{\teal}{\textcolor{teal}}
\newcommand{\zaff}{\textcolor{zaffre}}
\newcommand{\black}{\textcolor{black}}
\newcommand{\darkgreen}{\textcolor{DarkOliveGreen}}
\newcommand{\green}{\textcolor{OliveGreen}}

% Block coloring
\newenvironment{blockcolor}{\par \color{crimson} {\par}}

% \newcommand{\id}{\mathbf{id}\;}

% matrices -- vertical separators
\makeatletter
\renewcommand*\env@matrix[1][*\c@MaxMatrixCols{ c}]{%
  \hskip -\arraycolsep{}
  \let\@ifnextchar\new@ifnextchar{}
  \array{#1}}
\makeatother

% \usepackage{accode}
\usepackage{tikz}

% long multiplications
\usepackage{xlop}

% custom functions.
\renewcommand{\gcd}[2]{\mathbf{gcd}\;(#1,\;#2)}
\newcommand{\lcm}[2]{\mathbf{lcm}\;(#1,\;#2)}
\newcommand{\Therefore}{\therefore\;}
\newcommand{\However}{\dot{}\hspace{.045in}.\hspace{.045in} \dot{}\hspace{.095in}}

% resume includes
\usepackage[utf8]{inputenc}
\usepackage[full]{textcomp}
\usepackage{CJKutf8}
\usepackage[lf]{ebgaramond}

\usepackage[OT1]{fontenc}
\usepackage{enumitem}
\usepackage[scale=.75]{geometry}
\usepackage{url}

\pagestyle{headings}

\setlength\parindent{2em}

\thispagestyle{empty}

\newcommand{\cvsubsection}[1]{\subsection*{\hspace{1.45em}#1}}

\usepackage{enumitem}

% enumalph
\newenvironment{enumalph}{
  \begin{enumerate}[label=(\alph*)]
}{\end{enumerate}}

% enumroman
\newenvironment{enumroman}{
  \begin{enumerate}[label=(\roman*)]
}{\end{enumerate}}

\newenvironment{problab}[1]
{\noindent\textbf{Problem #1}.}
{\vskip 6pt}
\theoremstyle{remark}
\newtheorem*{solu}{Solution}

% import := definition
\usepackage{colonequals}

\usepackage{listings}
\usepackage{color}

% -- Defining colors:
\usepackage[dvipsnames]{xcolor}
\definecolor{codegreen}{rgb}{0,0.6,0}
\definecolor{codegray}{rgb}{0.5,0.5,0.5}
\definecolor{codepurple}{rgb}{0.58,0,0.82}
\definecolor{backcolour}{rgb}{0.95,0.95,0.92}
\definecolor{dkgreen}{rgb}{0,0.6,0}
\definecolor{gray}{rgb}{0.5,0.5,0.5}
\definecolor{mauve}{rgb}{0.58,0,0.82}

\lstset{frame=tb,
  backgroundcolor=\color{backcolour},   
  commentstyle=\color{codepurple},
  keywordstyle=\color{NavyBlue},
  numberstyle=\tiny\color{codegray},
  stringstyle=\color{codepurple},
  basicstyle=\ttfamily\footnotesize\bfseries,
  breakatwhitespace=false,         
  breaklines=true,                 
  captionpos=t,                    
  keepspaces=true,                 
  numbers=left,                    
  numbersep=5pt,                  
  showspaces=false,                
  showstringspaces=false,
  showtabs=false,                  
  tabsize=2,
  % escapeinside={\%*}{*)},          % if you want to add LaTeX within your code
}

\usepackage{booktabs}


\DeclareMathOperator{\Aut}{Aut}
\DeclareMathOperator{\opspan}{span}
\DeclareMathOperator{\Tr}{Tr}
\DeclareMathOperator{\Frac}{Frac}
\DeclareMathOperator{\ord}{ord}
\DeclareMathOperator{\Sym}{Sym}

\numberwithin{equation}{section}
\newtheorem{theorem}[equation]{Theorem}
\newtheorem{thm}[equation]{Theorem}
\newtheorem{lemma}[equation]{Lemma}
\newtheorem{lem}[equation]{Lemma}
\newtheorem{proposition}[equation]{Proposition}
\newtheorem{prop}[equation]{Proposition}
\newtheorem{corollary}[equation]{Corollary}
\newtheorem{claim}[equation]{Claim}
\newtheorem{cor}[equation]{Corollary}

\theoremstyle{definition}
\newtheorem{definition}[equation]{Definition}
\newtheorem{defn}[equation]{Definition}
\newtheorem{example}[equation]{Example}
\newtheorem{xca}[equation]{Exercise}
\newtheorem{notation}[equation]{Notation}
\theoremstyle{remark}
\newtheorem{remark}[equation]{Remark}

\numberwithin{equation}{section}


\usepackage[normalem]{ulem}
\usepackage{fullpage}
\usepackage{colonequals}
\usepackage{amssymb}
\usepackage{amsthm}
\usepackage{amsmath}
\usepackage{amsxtra}
\usepackage{mathtools}
\usepackage{mathrsfs}

\usepackage{hyperref}
\hypersetup{colorlinks=true,urlcolor=blue,citecolor=blue,linkcolor=blue}

\numberwithin{equation}{section}

\usepackage{amssymb}
\usepackage{amsthm}
\usepackage{amsmath}
\usepackage{amsxtra}

\setlength{\hfuzz}{4pt}

\DeclarePairedDelimiter{\abs}{\lvert}{\rvert}

\newcommand{\defi}[1]{\textsf{#1}} % for defined terms

\renewcommand{\baselinestretch}{1.5} 

\usepackage{titling}
\usepackage[english]{babel}
\usepackage[utf8]{inputenc}
\usepackage{amsmath, amsfonts, amsthm}
\usepackage{graphicx}
\usepackage[colorinlistoftodos]{todonotes}
\usepackage{subfig}
% \usepackage{mdframed} 
\usepackage{color}
\usepackage{tabu}
\usepackage{tikz}
\usepackage{enumerate}
\usepackage{multicol}
\usepackage{pgfplots}
\usepackage{csquotes}
\pgfplotsset{compat=1.18}
% \usepackage[style=iso]{datetime2}
\usepackage[mmddyyyy]{datetime}
\usepackage{multirow}
\usepackage{float}  % prevent table repositioning.
\usepackage{lipsum}


% SQL Symbols
\def\ojoin{\setbox0=\hbox{$\bowtie$}%
  \rule[-.02ex]{.25em}{.4pt}\llap{\rule[\ht0]{.25em}{.4pt}}}
\def\leftouterjoin{\mathbin{\ojoin\mkern-5.8mu\bowtie}}
\def\rightouterjoin{\mathbin{\bowtie\mkern-5.8mu\ojoin}}
\def\fullouterjoin{\mathbin{\ojoin\mkern-5.8mu\bowtie\mkern-5.8mu\ojoin}}

\newcounter{problem}
\setcounter{problem}{0}

\renewcommand{\theenumi}{\alph{enumi}}

\usepackage{MnSymbol}
\def \backmodels{\leftmodels}


\usepackage[english]{cleveref}

\def \vbar{\overline{v}}
\DeclareRobustCommand\Iff{\;\Longleftrightarrow\;}
\DeclareRobustCommand\iff{\;\leftrightarrow\;}
\DeclareRobustCommand\lto{\rightarrow}
\DeclareRobustCommand\backimplies{\Longleftarrow}
\newcommand{\bigland}{\bigwedge}

\DeclareRobustCommand{\step}{\vspace{0.5em}\noindent}

\DeclareRobustCommand{\set}[1]{\left\{#1\right\}}
\DeclareRobustCommand{\vec}[1]{\left\langle #1 \right\rangle}

\DeclareSymbolFont{CMsymbols}{OMS}{cmsy}{m}{n}
\SetSymbolFont{CMsymbols}{bold}{OMS}{cmsy}{b}{n}
\DeclareMathSymbol{\foralll}{\mathord}{CMsymbols}{"38}
\DeclareMathSymbol{\existss}{\mathord}{CMsymbols}{"39}
\renewcommand{\forall}{\foralll\;}
\renewcommand{\exists}{\existss\;}


\hyphenation{auto-morphism auto-morphisms homo-geneous}

%%%%%%%%%%%%%%%%%%%%%%%%%%%%%%%%%%%%%%%%%%%%%%%%%%%%%%%%%%%%%%%%%%%%%%
%%%%%%%%%%%%%%%%%%%%%%%%%%%%%%%%%%%%%%%%%%%%%%%%%%%%%%%%%%%%%%%%%%%%%%
%%%%%%%%%%%%%%%%%%%%%%%%%%%%%%%%%%%%%%%%%%%%%%%%%%%%%%%%%%%%%%%%%%%%%%

% \title{%
%   On the derivations and automorphisms\\
%   of the algebra $\kk\langle x,y\rangle/(yx-xy-x^N)$
% }
% \author{Amittai Siavava\thanks{%
%     Undergraduate student at Dartmouth College.
%   }
% }
% \ifoptionfinal
%   {\date{\today}}
%   {\date{Started on March 2021; compiled \today}}

  
\begin{document}

% \setlength{\headheight}{13.0pt}
% \setlength{\footskip}{13.0pt}


% TITLE
\newdate{due-date}{07}{02}{2024}
\PSET{5 --- \displaydate{due-date}}
  {Winter 2024}
  {Erchenko}
  {Amittai Siavava}
  {Math 63: Real Analysis}

% CREDIT STATEMENT
\CreditStatement{
  I worked on these problems alone,
  with reference to class notes and the following books:
  \begin{enumerate}
    \item \textbf{\textit{Introduction to Analysis}} by Maxwell Rosenlicht
  \end{enumerate}
}

% \bigskip

\begin{problem}
  If $a_1, a_2, a_3, \ldots$ is a bounded sequence of real numbers, define
  \begin{align*}
    \limsup\limits_{n \to \infty}{a_n}
    \colonequals
    \sup \set{x \in \R \given a_n > x \text{ for infinitely many } n \in \N}
    \\
    \liminf\limits_{n \to \infty}{a_n}
    \colonequals
    \inf \set{x \in \R \given a_n < x \text{ for infinitely many } n \in \N}
  \end{align*}

  Prove that $\liminf\limits_{n \to \infty}{a_n} \leq \limsup\limits_{n \to \infty}{a_n}$
  with the equality holding if and only if the sequence converges.
\end{problem}

\begin{answer}
  Let $A = a_1, a_2, a_3, \ldots$ be a bounded sequence of real numbers as defined above.

  \begin{claim}
    $\liminf\limits_{n \to \infty}{a_n} \leq \limsup\limits_{n \to \infty}{a_n}$
    \begin{proof}
      Let $L = \liminf\limits_{n \to \infty}{a_n}$ and $U = \limsup\limits_{n \to \infty}{a_n}$.
      By the definitions of $\liminf$ and $\limsup$, for any $\epsilon > 0$, there exist indices $n_1, n_2 \in \N$ such that
      \begin{align*}
        a_n &> L - \epsilon \quad \text{for infinitely many } n > n_1 \\
        a_n &< U + \epsilon \quad \text{for infinitely many } n > n_2
      \end{align*}
      Since $a_n > L - \epsilon$ for infinitely many $n$, and $a_n < U + \epsilon$ for infinitely many $n$,
      it implies that $L - \epsilon < U + \epsilon$.
      Since this holds for any $\epsilon > 0$, we conclude that $L \leq U$.
    \end{proof}
  \end{claim}

  \newpage
  \begin{claim}
    If the sequence converges, then $\liminf\limits_{n \to \infty}{a_n} = \limsup\limits_{n \to \infty}{a_n}$.

    \begin{proof}
      Suppose the sequence $A = a_1, a_2, a_3, \ldots$ converges to some point $a$. \\
      Then, for any $\epsilon > 0$, there exists $N \in \N$ such that
      $d(a_n, a) < \epsilon$ for all $n \geq N$,
      so that $a - \epsilon < a_n < a + \epsilon$.
      Therefore, $a - \epsilon$ is a lower bound of a set containing
      infinitely many points of the sequence,
      and $a + \epsilon$ is an upper bound of a set containing
      infinitely many points of the sequence.
      By definition of $\liminf$ and $\limsup$,
      $\liminf\limits_{n \to \infty}{a_n} \geq a - \epsilon$
      since it has to be less than or equal to any lower bound of infinitely
      many points of the sequence. \\
      Similarly, $\limsup\limits_{n \to \infty}{a_n} \leq a + \epsilon$.\\
      Therefore,
      \[ a - \epsilon \leq \liminf\limits_{n \to \infty}{a_n} \leq \limsup\limits_{n \to \infty}{a_n} \leq a + \epsilon \]
      for any $\epsilon > 0$.
      By making $\epsilon$ arbitrarily
      small---which we can, since $A$ converges---we conclude that \\
      $\liminf\limits_{n \to \infty}{a_n} = \limsup\limits_{n \to \infty}{a_n} = a$.
    \end{proof}
  \end{claim}
  
\end{answer}


\newpage
\begin{problem}
  Prove that any bounded open subset of $\R$
  is the union of disjoint open intervals.
\end{problem}

\begin{answer}

\end{answer}


\newpage
\begin{problem}
  Use Taylor's Theorem to prove the ``binomial theorem'' for $n \in \N$:
  \[ (a + x)^n = a^n + na^{n-1}x + \frac{n(n-1)}{2}a^{n-2}x^2 + \cdots + x^n. \]
\end{problem}

\begin{answer}
  We will use Taylor's Theorem to prove the binomial theorem for $n \in \N$.
  Let $f : \R \to \R$ with $f(x) = x^n$.

  \begin{enumarabic}
    \item First, consider the value of $f$ at $x = a$:
      \[ f(a) = a^n \]
    \item Next, consider the derivatives of $f$:
      \begin{align*}
        f'(x) &= n(x)^{n-1} &= \frac{n}{(n-1)!} (x)^{n-1} \\
        f''(x) &= n(n-1)(x)^{n-2} &= \frac{n}{(n-2)!} (x)^{n-2} \\
        f'''(x) &= n(n-1)(n-2)(x)^{n-3} &= \frac{n}{(n-3)!} (x)^{n-3} \\
        &\vdots \\
        f^{(k)}(x) &= n(n-1)\cdots(n-k+1)(x)^{n-k} &= \frac{n!}{(n-k)!} (x)^{n-k} \\
        &\vdots \\
        f^{(n)}(x) &= n! &= n!
      \end{align*}
    \item Now, consider the Taylor expansion of $f(x)$ about a point $a \in \R$:
      \[ f(x) = f(a) + f'(a)(x - a) + \frac{f''(a)}{2!}(x - a)^2 + \cdots \]
    
    \item Plugging in $f = (x)^n$ into the Taylor expansion gives,
      with center $a$ and point $a + x$:
      \[
        f(a+x) = (a+x)^n = a^n + na^{n-1}x + \frac{n(n-1)}{2}(a + a)^{n-2}x^2
              + \cdots + x^n,
      \]
      which is the binomial theorem for $n \in \N$.
  \end{enumarabic}
\end{answer}


\newpage
\begin{problem}
  Let $A = \set{1, 2, \ldots, n}$.
  \begin{enumalph}
    \item Show that the cardinality of $\calP(A)$ is $2^n$.
      \begin{answer}
        \textbf{For simplicity, let's use the notation
        $A_n$ to denote the set $\set{1, 2, \ldots, n}$.}
        \begin{proof}
          This is a proof by induction on $n$.

          \begin{enumroman}
            \item \emph{Base Case}: $n = 0$\\
              We show that $\abs{\calP(A_0)} = 2^0 = 1$.
              Note that $A_0 = \emptyset$, since $A_0$ contains
              zero elements. The only possible subset is the entire set
              itself, so $\abs{\calP(A_0)} = 2^0 = 1$.
            \item \emph{Induction Hypothesis}: \\
              Assume that $\abs{\calP(A_n)} = 2^n$,
              we show that $\abs{\calP(A_{n+1})} = 2^{n+1}$.
            \item \emph{Inductive Step}:\\
              How can we construct $\calP(A_{n+1})$ from $\calP(A_n)$?
              An element in $\calP(A_{n+1})$ is either an element in
              $\calP(A_n)$, or it is an addition of the element $(n+1)$
              to an element in $\calP(A_n)$. Thus; 
              \[
                \calP(A_{n+1}) = \calP(A_n) \cup \set{a \cup \set{(n+1)}\, |\; a \in \calP(A_n)}
              \]
              However, $\calP(A_n)$ and $\set{a \cup \set{(n+1)}\, |\; a \in \calP(A_n)}$
              are disjoint, since each element in \\
              \crim{$\set{a \cup \set{(n+1)}\, |\; a \in \calP(A_n)}$} contains
              the element $(n+1)$, but no element in $\calP(A_n)$ contains the element $(n+1)$.
              Therefore;
              \begin{align*}
                \abs{\calP(A_{n+1})} &= \abs{\calP(A_n)} + \abs{\set{a \cup \set{(n+1)}\, |\; a \in \calP(A_n)}} \\
                                                &= \abs{\calP(A_n)} + \abs{\calP(A_n)} \\
                                                &= 2 \times \abs{\calP(A_n)} \\
                                                &= 2 \cdot 2^n \\
                                                &= 2^{n+1}
              \end{align*}
          \end{enumroman}
        \end{proof}
      \end{answer}
    \item How many functions are there from $A$ to $A$? Explain.
      \begin{answer}
        A function sends each point in the domain to exactly one point in the
        codomain. Since there are $n$ points in the domain
        and $n$ points in the codomain, there are $n$ choices for each point
        to go to, so \textbf{there are $n^n$ functions from $A$ to $A$.}
        However, most of these functions will be neither injective nor surjective!

      \end{answer}
    \item How many onto (surjective) functions are there from $A$ to $A$? Explain.
      \begin{answer}
        A function is surjective if every point in the codomain is hit by
        some point in the domain. In this case, the codomain and the domain
        are the same set, so surjectivity also implies injectivity.
        Thus; there are $n$ possible points for the first point to go to,
        then $n-1$ possible points for the second point to go to
        (since it can no longer map to the same point as the first point),
        and $n-2$ possible points for the third point to go to, and so on.
        \textbf{In total, there are $n!$ surjective functions from $A$ to $A$.}
      \end{answer}
  \end{enumalph}
\end{problem}


\newpage
\begin{problem}
  Let $(E, d_E)$ be a compact metric space, and
  let $f, f_1, f_2, f_3, \ldots : E \to \R$ be continuous real-values functions on $E$,
  with $\displaystyle \lim_{n \to \infty} f_n = f$.
  Prove that if $f_1(p) \leq f_2(p) \leq f_3(p) \leq \cdots$ for all $p \in E$
  then the sequence $f_1, f_2, f_3, \ldots$ converges uniformly.
\end{problem}

\begin{answer}
  We will show that the sequence $f_1, f_2, f_3, \ldots$ converges uniformly to $f$.
  Let $\epsilon > 0$.
  We need to find an $N \in \N$ such that for all $n \geq N$ and all $p \in E$,
  $\abs{f_n(p) - f(p)} < \epsilon$.

  \begin{claim}
    There exists an $N \in \N$ such that for all $n \geq N$ and all $p \in E$,
    $\abs{f_n(p) - f(p)} < \epsilon$.
    \begin{proof}
      Since $f_n \to f$, for every $p \in E$, there exists an $N_p \in \N$
      such that for all $n \geq N_p$, $\abs{f_n(p) - f(p)} < \frac{\epsilon}{3}$.
      Since $E$ is compact, there exists a finite subcover $\set{U_1, U_2, \ldots, U_k}$
      of $E$.
      Let $N = \max \set{N_{p_1}, N_{p_2}, \ldots, N_{p_k}}$.
      Then, for all $n \geq N$ and all $p \in E$, $\abs{f_n(p) - f(p)} < \epsilon$.
    \end{proof}
  \end{claim}
  Thus, the sequence $f_1, f_2, f_3, \ldots$ converges uniformly to $f$.
\end{answer}


\newpage
\begin{problem}
  Let $(X, d_X)$ and $(Y, d_Y)$ be metric spaces.
  Assume $(Y, d_Y)$ is complete.
  Show that a sequence of functions $f_n : X \to Y$ converges uniformly
  on $X$ if and only if it is uniformly Cauchy on $X$.
\end{problem}

\begin{answer}
  We will show that a sequence of functions $f_n : X \to Y$ converges uniformly
  on $X$ if and only if it is uniformly Cauchy on $X$.

  \begin{claim}
    The sequence of functions $f_n : X \to Y$ converges uniformly on $X$
    if and only if it is uniformly Cauchy on $X$.
    \begin{proof}
      We will first show that if the sequence of functions $f_n : X \to Y$
      converges uniformly on $X$, then it is uniformly Cauchy on $X$.
      We will then show that if the sequence of functions $f_n : X \to Y$
      is uniformly Cauchy on $X$, then it converges uniformly on $X$.
      \begin{enumarabic}
        \item $(\Longrightarrow)$
        Suppose the sequence of functions $f_n : X \to Y$ converges uniformly on $X$.
        Then, for every $\epsilon > 0$, there exists an $N \in \N$ such that
        for all $n, m \geq N$ and all $x \in X$, $\abs{f_n(x) - f_m(x)} < \epsilon$.
        Thus, for all $n, m \geq N$,
        \[ \displaystyle \sup_{x \in X} \abs{f_n(x) - f_m(x)} < \epsilon. \]
        Therefore, the sequence of functions $f_n : X \to Y$ is uniformly Cauchy on $X$.
  
        \step
        \item $(\Longleftarrow)$
        Suppose the sequence of functions $f_n : X \to Y$ is uniformly Cauchy on $X$.
        Then, for every $\epsilon > 0$, there exists an $N \in \N$ such that
        for all $n, m \geq N$,
        \[ \sup_{x \in X} \abs{f_n(x) - f_m(x)} < \epsilon. \]
        Since $(Y, d_Y)$ is complete, the sequence of functions $f_n : X \to Y$
        converges pointwise to some function $f : X \to Y$.
        We need to show that the sequence converges uniformly on $X$.
        Meaning, for $\epsilon > 0$.
        We need to find an $N \in \N$ such that for all $n \geq N$ and all $x \in X$,
        $\abs{f_n(x) - f(x)} < \epsilon$.
  
        \step
        \begin{enumarabic}
          \item
            Since the sequence of functions $f_n : X \to Y$ is uniformly Cauchy on $X$,
            there exists an $N \in \N$ such that for all $n, m \geq N$,
            $\displaystyle \sup_{x \in X} \abs{f_n(x) - f_m(x)} < \frac{\epsilon}{2}$.
          
          \item
            Furthermore, since the sequence of functions $f_n : X \to Y$ converges
            pointwise to $f : X \to Y$,
            there exists an $N' \in \N$ such that for all $n \geq N'$,
            $\abs{f_n(x) - f(x)} < \frac{\epsilon}{2}$ for all $x \in X$.

          \item
            Let $N'' = \max \set{N, N'}$.
            Then, for all $n \geq N''$ and all $x \in X$,
        \end{enumarabic}
        \begin{align*}
          \abs{f_n(x) - f(x)}
            &\leq \abs{f_n(x) - f_{N''}(x)} + \abs{f_{N''}(x) - f(x)} \\
            &\qquad < \frac{\epsilon}{2} + \frac{\epsilon}{2} = \epsilon.
        \end{align*}
      \end{enumarabic}
      Thus, the sequence of functions $f_n : X \to Y$ converges uniformly on $X$.
    \end{proof}
  \end{claim}
\end{answer}


\vfill

\end{document}
