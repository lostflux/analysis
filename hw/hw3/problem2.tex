\begin{problem}
  Prove that any bounded open subset of $\R$
  is the union of disjoint open intervals.
\end{problem}

\begin{answer}
  Let $E \subseteq \R$ be a bounded open subset of $\R$.
  Since $E$ is bounded \emph{and} open, there exists some $a, b \in \R$ such that
  $a < x < b$ for all $x \in E$.
  Thus, $E \subseteq (a, b)$.\\
  For each element $x \in E$, let $I_x = (a_x, b_x)$ be the largest
  \emph{continuous} open interval around $x$ that is contained entirely in $E$.

  \begin{claim}~\label{claim:2.2}
    For any two such intervals $I_x$ and $I_y$,
    either $I_x = I_y$ or $I_x \cap I_y = \emptyset$.

    \begin{proof}
      Since we pick $I_x$ and $I_y$ to be the largest open intervals
      around $x$ and $y$ respectively that are contained in $E$,
      if $I_x \cap I_y \neq \emptyset$, then there exists some element
      $z$ in both $I_x$ and $I_y$. If $I_z$ is the maximal interval
      around $z$, then every element in
      $I_x$ is in $I_z$ and every element in $I_y$ is in $I_z$.
      Likewise, every element in $I_z$ is in both $I_x$ and $I_y$
      (since $x$ is in both $I_x$ and $I_y$).
      Therefore, $I_x = I_y = I_z$.
    \end{proof}
  \end{claim}

  \begin{claim}~\label{claim:2.3}
    $E = \bigcup\limits_{x \in E} I_x$.

    \begin{proof}
      We shall show that the two sets are equal using double containment.
      \begin{enumarabic}
        \item $E \supseteq \bigcup_{x \in E} I_x$:\\
          Let $a \in \bigcup_{x \in E} I_x$ be arbitrary,
          with $I_a$ as the largest open interval around $a$ that is contained in $E$,
          By definition of $I_a$, every point in $I_a$ is contained in $E$,
          so $a \in E$.
        \item $E \subseteq \bigcup\limits_{x \in E} I_x$:\\
          Let $a \in E$ be arbitrary, with $I_a$ being the
          largest open interval around $x$ that is contained in $E$.
          By definition of $I_a$, $a \in I_a$, so $a \in \bigcup\limits_{x \in E} I_x$.
      \end{enumarabic}
    \end{proof}
    
  \end{claim}

  \begin{claim}
    Any bounded open subset of $\R$ is the union of disjoint open intervals.

    \begin{proof}
      By connecting claims~\ref{claim:2.2} and~\ref{claim:2.3},
      we have that $E = \bigcup\limits_{x \in E} I_x$,
      where each $I_x$ is a disjoint open interval.
    \end{proof}
  \end{claim}
\end{answer}

\crim{
  I attempted an alternative proof on the next page.
  I would love some feedback on whether it is correct or not,
  but if you only have to grade one then please grade the first one.
}

\newpage
I wasn't sure if the previous technique is correct,
so I attempted an alternative construction.

\begin{answer}
  \begin{claim}
    Any bounded open subset of $\R$ is the union of disjoint open intervals.

    \begin{proof}
      Let $E \subseteq \R$ be a bounded open subset of $\R$.
      Since $E$ is bounded \emph{and} open, there exists some $a, b \in \R$ such that
      $a < x < b$ for all $x \in E$.
      Thus, $E \subseteq (a, b)$.
    
      \begin{enumroman}
        \item If $E = \emptyset$, then it is the union of disjoint open intervals
          since the empty set is trivially open.
        \item If $E = (a, b)$, then it is the union of the disjoint open intervals
          since $(a, b)$ is open.
        \item If $E \neq (a, b)$, then there exists some $c \in (a, b)$ such that
          $c \not \in E$. Write $E = E_1 \cup E_2$ with $E_1 \subseteq (a, c)$
          and $E_2 \subseteq (c, b)$.
          $E_1$ and $E_2$ are disjoint since $a < c < b$, and $c \not \in E_1$,
          $c \not \in E_2$.
          \begin{itemize}
            \item If $E_1 = (a, c)$ and $E_2 = (c, b)$, then we are done
              since $(a, c)$ and $(c, b)$ are both open.
            \item If $E_1 \neq (a, c)$ or $E_2 \neq (c, b)$,
              repeat the process of splitting $E_1$ and/or $E_2$ into smaller open intervals
              until equality.
          \end{itemize}
          We can then write $E = E_i \cup E_j \cup \dots$ for some
          open intervals $E_i, E_j, \dots$ contained in $(a, b)$.
      \end{enumroman}
    \end{proof}
  \end{claim}
\end{answer}
