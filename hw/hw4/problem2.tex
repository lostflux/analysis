\begin{problem}
  Let $\displaystyle x_n = \parens{1 + \frac{1}{n}}^n$
  for all $n \in \N$.

  \begin{remark}~\label{remark:euler}
    The Euler number $e$ can be defined as
    $\displaystyle \lim_{n \to \infty}{\parens{1 + \frac{1}{n}}^n}$.
  \end{remark}

  \begin{enumroman}
    \item~\label{step:2.1} Using induction, show that for all $x > -1$ and $n \in \N$, we have
      \[ \parens{1 + x}^n \geq 1 + nx. \]

      \begin{answer}
        We shall prove the claim by induction on $n$.

        \begin{proof}
          Let $x > -1$ and $n \in \N$.
          \begin{enumerate}
            \item Base case: $n = 1$.
              \begin{alignat*}{5}
                &\parens{1 + x}^n &&= \parens{1 + x}^1 &&= 1 + x &&= 1 + 1 \cdot x &&= 1 + nx
              \end{alignat*}
              Since $1 + nx \geq 1 + nx$, the inequality holds for $n = 1$.
            \item Inductive step: Suppose $\parens{1 + x}^n \geq 1 + nx$.
              \begin{alignat*}{5}
                &\parens{1 + x}^{n+1} &&= \crim{\parens{1 + x}^n} \parens{1 + x} \\
                & &&\geq \crim{\parens{1 + nx}} \parens{1 + x} &&= 1 + x + nx + nx^2 \\
                & && &&\geq 1 + x + nx &&= 1 + (n+1)x
              \end{alignat*}
          \end{enumerate}
          Therefore, $\parens{1 + x}^n \geq 1 + nx$ for all $n \in \N$.
        \end{proof}
      \end{answer}

    \newpage
    \item~\label{step:2.2} Using the previous item, show that
      $\displaystyle  \frac{x_{n+1}}{x_n} \geq 1$ so $x_n$
      is monotonically increasing.

      \begin{answer}
        To show this, we need to show that $x_{n+1} \geq x_n$ for all $n \in \N$.
        \begin{alignat*}{5}
          &\frac{x_{n+1}}{x_n} &&= \frac{\parens{1 + \frac{1}{n+1}}^{n+1}}{\parens{1 + \frac{1}{n}}^n} \\
          & &&= \frac{\parens{1 + \frac{1}{n+1}}^{n}}{\parens{1 + \frac{1}{n}}^n} \cdot \parens{1 + \frac{1}{n+1}} \\
          & &&= \parens{\frac{1 + \frac{1}{n+1}}{1 + {\frac{1}{n}}}}^n \cdot \parens{1 + \frac{1}{n+1}} \\
          & &&= \parens{\frac{1 + \frac{1}{n+1}}{\frac{n+1}{n}}}^n \cdot \parens{1 + \frac{1}{n+1}} \\
          & &&= \parens{\frac{n + \frac{n}{n+1}}{n+1}}^n \cdot \parens{1 + \frac{1}{n+1}} \\
          & &&= \parens{\frac{n}{n+1} + \frac{n}{\parens{n+1}^2}}^n \cdot \parens{1 + \frac{1}{n+1}} \\
          & &&= \parens{1 - \frac{1}{n+1} + \frac{n}{\parens{n+1}^2}}^n \cdot \parens{1 + \frac{1}{n+1}} \\
          & &&= \parens{1 - \frac{-\parens{n+1} + n}{\parens{n+1}^2}}^n \cdot \parens{1 + \frac{1}{n+1}} \\
          & &&= \crim{\parens{1 - \frac{1}{\parens{n+1}^2}}^n} \cdot \parens{1 + \frac{1}{n+1}} \\
          & &&\qquad \geq \crim{\parens{1 - \frac{n}{\parens{n+1}^2}}} \cdot \parens{1 + \frac{n}{n+1}} \qquad \text{(by the previous proof)} \\
          & &&\qquad \qquad >\parens{1 - \frac{n+1}{\parens{n+1}^2}} \cdot \parens{1 + \frac{n}{n+1}} \qquad \text{(subtracting a bigger term)} \\
          & &&\qquad \qquad = \parens{1 - \frac{1}{n+1}} \cdot \parens{1 + \frac{n}{n+1}} \\
          & &&\qquad \qquad = 1 - \frac{1}{n+1} + \frac{1}{n} - \frac{1}{n\parens{n+1}} \\
          & &&\qquad \qquad = \frac{n\parens{n+1} - n + \parens{n+1} - 1}{n\parens{n+1}} \\
          & &&\qquad \qquad = \frac{n\parens{n+1}}{n\parens{n+1}} \\
          & &&\qquad \qquad = 1
        \end{alignat*}
      \end{answer}

    \newpage
    \item~\label{step:2.3} Show that $x_n$ is bounded using the binomial formula
      \[
        \parens{a + b}^n =
        \frac{n!}{k! (n-k)!} a^{n-k} b^k =
        \sum_{k=0}^n{\binom{n}{k} a^{n-k} b^k}.
      \]
  
      \begin{answer}
        Fix $a$ and $b$ to arbitrary real numbers.
        We shall prove this by induction on $n$.
        \begin{enumalph}
          \item Base case: $n = 1$
            \begin{alignat*}{5}
              & \parens{a + b}^n &&= a + b \\
              & &&= \binom{1}{0} a^{1} \cdot b^0 + \binom{1}{1} a^0 b^1 \\
              & &&= \binom{1}{0} a^{1-0} b^0 + \binom{1}{1} a^{1-1} b^1 &&= \sum_{k=0}^1{\binom{n}{k} a^{n-k} b^k}
            \end{alignat*}
          \item Inductive step: Assume the invariant holds
            for ${(a+b)}^n$.
            We shall show that it holds for ${(a+b)}^{(n+1)}$.
            \begin{alignat*}{5}
              & \parens{a + b}^{n+1} &&= \parens{a + b}^n \parens{a + b} \\
              & &&= \parens{\sum_{k=0}^n{\binom{n}{k} a^{n-k} b^k}} \parens{a + b} \\
              & &&= \sum_{k=0}^n{\binom{n}{k} a^{n-k} b^k} \cdot a + \sum_{k=0}^n{\binom{n}{k} a^{n-k} b^k} \cdot b \\
              & &&= \sum_{k=0}^n{\binom{n}{k} a^{n+1-k} b^k} + \sum_{k=0}^n{\binom{n}{k} a^{n-k} b^{k+1}} \\
              & &&= \sum_{k=0}^n{\binom{n}{k} a^{n+1-k} b^k + a^{n-k} b^{k+1}} \\
              & &&= \sum_{k=0}^n{\binom{n}{k} a^{n+1-k} b^k + \binom{n}{k-1} a^{n+1-k} b^k} \qquad \text{(Grouping together equal powers)} \\
              & &&= \sum_{k=0}^n{\parens{\frac{n!}{k! (n-k)!} + \frac{n!}{(k-1)! (n-k+1)!}}} a^{n+1-k} b^k \\
              & &&= \sum_{k=0}^n{\parens{\frac{\parens{n+1-k}n! + kn!}{k!\parens{n+1-k}!}}} a^{n+1-k} b^k \\
              & &&= \sum_{k=0}^n{\parens{\frac{\parens{n+1}n!}{k!\parens{n+1-k}!}}} a^{n+1-k} b^k \\
              & &&= \sum_{k=0}^n{\parens{\frac{\parens{n+1}!}{k!\parens{n+1-k}!}}} a^{n+1-k} b^k \\
              & &&= \sum_{k=0}^n{\binom{n+1}{k} a^{n+1-k} b^k} \\
              & &&= \sum_{k=0}^{n+1}{\binom{n+1}{k} a^{n+1-k} b^k}
            \end{alignat*}
        \end{enumalph}
        Since $a$ and $b$ were arbitrary, this holds for all $a$ and $b$,
        including $a = 1$ and $b = \frac{1}{n}$, thus
        $x_n$ is bounded by the binomial formula since equality
        satisfies both \emph{less than or equal to} and \emph{greater than or equal to}.
      \end{answer}

    \item~\label{step:2.4} Show that $\braces{x_n}$ is convergent.
    \begin{answer}
      As seen in \ref{step:2.2} and \ref{step:2.3},
      $\braces{x_n}$ is monotonically increasing and is bounded above.

      \begin{claim}
        $\braces{x_n}$ is convergent.

        \begin{proof}
          Suppose $\braces{x_n}$ is not convergent.
          Since $\braces{x_n}$ is monotonically increasing, this implies that
          it must not be bounded above and
          $\displaystyle \lim_{n \to \infty} x_n = \infty$,
          which contradicts the known fact (by remark \ref{remark:euler}) that
          $\displaystyle \lim_{n \to \infty} x_n = e \neq \infty$.
          Therefore, the sequence must be convergent.
        \end{proof}
      \end{claim}
    \end{answer}
  \end{enumroman}
\end{problem}
