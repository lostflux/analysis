\begin{problem}
  Let $U, V$ be open intervals in $\R$,
  and let $f : U \to V$ be a function that is strictly increasing
  (i.e. if $x, y \in U$ and $x < y$, then $f(x) < f(y)$) and onto.
  Prove that $f$ and $f^{-1}$ are continuous.

  \begin{answer}
    \begin{claim}
      $f$ is continuous.
      \begin{proof}
        We need to show that for every $x \in U$ and every $\epsilon > 0$,
        there exists $\delta > 0$ such that for all $u \in U$, if $d(u, x) < \delta$
        then $d(f(u), f(x)) < \epsilon$.

        \step
        For any $x \in U$, let $x' \in V$ such that $x' = f(x)$. \\
        For any $\epsilon > 0$, pick $a', b' \in V$ such that
        $x' - \epsilon = a' < x' < b' = x' + \epsilon$. \\
        Since $f$ is onto, there exists $a, b \in U$ such that $f(a) = a'$ and $f(b) = b'$.
        Furthermore, since $f$ is strictly increasing, $a' < x' < b'$ implies $a < x < b$.
        Set $\delta = \min\set{x - a, b - x}$. Then, for any $u \in U$,
        if $d(u, x) < \delta$, then $a < u < b$, meaning $a' < f(u) < b'$ (since $f$ is strictly increasing).
        This means $d(f(u), x') < \epsilon$, since $(a', b')$ is the open interval or radius $\epsilon$
        around $x'$. \\
        Thus, $f$ is continuous.
      \end{proof}
    \end{claim}

    \begin{claim}
      $f^{-1}$ is continuous.
      \begin{proof}
        We need to show that for every $y \in V$ and every $\epsilon > 0$,
        there exists $\delta > 0$ such that for all $v \in V$, if $d(v, y) < \delta$
        then $d(f^{-1}(v), f^{-1}(y)) < \epsilon$.

        \step
        For any $y' \in V$, let $y \in U$ such that $y = f^{-1}(y')$ --- (or, $y' = f(y)$). \\
        For any $\epsilon > 0$, pick $a, b \in U$ such that
        $y - \epsilon = a < y < b = y + \epsilon$. \\
        Let $a', b' \in V$ such that $f(a) = a'$ and $f(b) = b'$.
        Since $f$ is strictly increasing, $a < y < b$ implies $a' < y' < b'$.
        Set $\delta = \min\set{y' - a', b' - y'}$. Then, for any $v \in V$, let $u \in U$ such that $v = f(u)$ and $u = f^{-1}(v)$.
        Then, if $d(v, y') < \delta$, then $a' < v' < b'$, meaning $a < f^{-1}(v') < b$ (since $f$ is strictly increasing).
        This means $d(f^{-1}(v), y') < \epsilon$, since $(a, b)$ is the open interval or radius $\epsilon$
        around $y'$. \\
        Thus, $f^{-1}$ is continuous.
      \end{proof}
    \end{claim}
  \end{answer}
\end{problem}
