\begin{problem}
  Prove that every bounded sequence of real numbers has a convergent subsequence
  (This statement is known as the \emph{Bolzano-Weierstrass Theorem}).

  \emph{
    [Hint]: Construct a Cauchy subsequence from the given sequence
    by constructing a sequence of nested intervals whose length converges to $0$
    and each  interval has infinitely many elements from the original sequence.
  }
\end{problem}
\begin{answer}
  Let $P \colonequals \braces{p_n}_1^\infty \subseteq \R$ be a bounded sequence of real numbers.
  Then $P$ is contained in some interval $[a, b]$.
  For an interval $S$, let $\calL(S)$ denote the length of the interval.
  We shall construct a sequence of nested intervals as follows:
  \begin{enumarabic}
    \item Let $I_0 \colonequals [a, b]$. Note that $\calL(I_0) = \abs{a - b}$.
    \item Let $I_1$ be whichever of
      $\displaystyle \brackets{a, \frac{a+b}{2}}$
      and $\displaystyle \brackets{\frac{a+b}{2}, b}$
      contains infinitely many elements of $P$
      at least one of them must contain infinitely many elements of $P$,
      since their union is $I_0$, which contains infinitely many elements of $P$.
      If both contain infinitely many elements of $P$, pick either one.
      Note that $\displaystyle \calL(I_1) = \frac{\abs{a - b}}{2}$ and $I_1 \subseteq I_0$.
    \item Construct $I_2$ to be whichever half of $I_1$
      contains infinitely many elements of $P$
      such that $\displaystyle \calL(I_2) = \frac{\abs{a - b}}{2^2}$ and $I_2 \subseteq I_1$.
    \item For each $n \in \N$, recursively construct $I_n$ to be whichever half of $I_{n-1}$
      contains infinitely many elements of $P$
      so that $\displaystyle \calL(I_n) = \frac{\abs{a - b}}{2^n}$
      and $I_n \subseteq I_{n-1} \subseteq \ldots \subseteq I_0$.
  \end{enumarabic}
  Next, for each $n \in \N$, pick any element $q_n$ from the sequence $P$
  such that $q_n \in I_n$.

  \begin{claim}
    $\braces{q_n}$ is Cauchy.
    \begin{proof}
      For any $\epsilon > 0$.
      Since $\displaystyle \lim_{n \to \infty}{\calL(I_n)} = 0$,
      there exists $N \in \N$ such that $\displaystyle \calL(I_N) < \epsilon$.
      Note that for any $n, m > N$, $q_m \in I_N$ and $q_n \in I_N$. Therefore,
      \[ 
        d(q_m, q_n) \leq \calL(I_N) < \epsilon
      \]
    \end{proof}
  \end{claim}

  \step
  Thus, $\braces{q_n}_1^\infty$ is Cauchy,
  and $\braces{q_n}_1^\infty \subseteq \brackets{a, b}$,
  so $\braces{q_n}_1^\infty$ is bounded.
  By the completeness of $\R$, $\braces{q_n}_1^\infty$ converges.
\end{answer}
