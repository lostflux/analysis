\begin{problem}
  Prove that a function is invertible if and only if it is a bijection.
\end{problem}

\begin{answer}
  \begin{definition}
    A function $f : A \to B$ is \emph{invertible} if there exists a function
    $g : B \to A$ such that $g \circ f = \id_A$ and $f \circ g = \id_B$.
  \end{definition}

  \step
  For the sake of contradiction, suppose that $f : A \to B$ is invertible
  but not a bijection. Let $g : B \to A$ be its inverse.
   \begin{enumroman}
    \item If $f$ is not a bijection, then either it is not an injection or
      it is not a surjection (or both!).
    \item Suppose $f$ is not an injection.
      Then there exists some $a_1, a_2 \in A$,
      $a_1 \neq a_2$, and some $b \in B$ such that $f(a_1) = f(a_2) = b$.
      Since $g$ is the inverse of $f$, we have $g(f(x)) = x$ for all $x \in A$,
      which implies that $g(b) = a_1$ and $g(b) = a_2$.
      But $a_1 \neq a_2$, implying that $g$ maps $b$ to two different elements
      hence $g$ is not a function.
      Therefore, any apparent inverse of $f$ is not a function, contradicting
      the assumption that $f$ is invertible.~\label{step:6.1}

    \item Suppose $f$ is not a surjection.
      Then there exists some $b \in B$ such that 
      $f(a) \neq b$ for all $a \in A$.
      However, $g$, the inverse of $f$, acts from $B$ to $A$,
      so it must assign some element $a' \in A$ to $b$.
      This implies that $g(b) = a'$ for some $a' \in A$.
      But $g$ was taken to be the inverse of $f$, so it must map
      $f(a') = a'$, where $f(a') \neq b$ (since we picked $b$ to be outside the image of $f$).
      Therefore, $g$ maps two different elements of $B$ to the same element $a' \in A$,
      so $g$ is not injective.
      From the previous result in~\ref{step:6.1}, this implies that $g$ is not invertible.
      If $g$ is not invertible then it cannot be the inverse of $f$,
      so $f$ is also not invertible.
    
   \end{enumroman}


\end{answer}
