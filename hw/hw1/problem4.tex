\begin{problem}
  Let $A = \set{1, 2, \ldots, n}$.
  \begin{enumalph}
    \item Show that the cardinality of $\calP(A)$ is $2^n$.
      \begin{answer}
        \textbf{For simplicity, let's use the notation
        $A_n$ to denote the set $\set{1, 2, \ldots, n}$.}
        \begin{proof}
          This is a proof by induction on $n$.

          \begin{enumroman}
            \item \emph{Base Case}: $n = 0$\\
              We show that $\abs{\calP(A_0)} = 2^0 = 1$.
              Note that $A_0 = \emptyset$, since $A_0$ contains
              zero elements. The only possible subset is the entire set
              itself, so $\abs{\calP(A_0)} = 2^0 = 1$.
            \item \emph{Induction Hypothesis}: \\
              Assume that $\abs{\calP(A_n)} = 2^n$,
              we show that $\abs{\calP(A_{n+1})} = 2^{n+1}$.
            \item \emph{Inductive Step}:\\
              How can we construct $\calP(A_{n+1})$ from $\calP(A_n)$?
              An element in $\calP(A_{n+1})$ is either an element in
              $\calP(A_n)$, or it is an addition of the element $(n+1)$
              to an element in $\calP(A_n)$. Thus; 
              \[
                \calP(A_{n+1}) = \calP(A_n) \cup \set{a \cup \set{(n+1)}\, |\; a \in \calP(A_n)}
              \]
              However, $\calP(A_n)$ and $\set{a \cup \set{(n+1)}\, |\; a \in \calP(A_n)}$
              are disjoint, since each element in \\
              \crim{$\set{a \cup \set{(n+1)}\, |\; a \in \calP(A_n)}$} contains
              the element $(n+1)$, but no element in $\calP(A_n)$ contains the element $(n+1)$.
              Therefore;
              \begin{align*}
                \abs{\calP(A_{n+1})} &= \abs{\calP(A_n)} + \abs{\set{a \cup \set{(n+1)}\, |\; a \in \calP(A_n)}} \\
                                                &= \abs{\calP(A_n)} + \abs{\calP(A_n)} \\
                                                &= 2 \times \abs{\calP(A_n)} \\
                                                &= 2 \cdot 2^n \\
                                                &= 2^{n+1}
              \end{align*}
          \end{enumroman}
        \end{proof}
      \end{answer}
    \item How many functions are there from $A$ to $A$? Explain.
      \begin{answer}
        A function sends each point in the domain to exactly one point in the
        codomain. Since there are $n$ points in the domain
        and $n$ points in the codomain, there are $n$ choices for each point
        to go to, so \textbf{there are $n^n$ functions from $A$ to $A$.}
        However, most of these functions will be neither injective nor surjective!

      \end{answer}
    \item How many onto (surjective) functions are there from $A$ to $A$? Explain.
      \begin{answer}
        A function is surjective if every point in the codomain is hit by
        some point in the domain. In this case, the codomain and the domain
        are the same set, so surjectivity also implies injectivity.
        Thus; there are $n$ possible points for the first point to go to,
        then $n-1$ possible points for the second point to go to
        (since it can no longer map to the same point as the first point),
        and $n-2$ possible points for the third point to go to, and so on.
        \textbf{In total, there are $n!$ surjective functions from $A$ to $A$.}
      \end{answer}
  \end{enumalph}
\end{problem}
