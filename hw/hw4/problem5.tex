\begin{problem}
  Let $(X, d)$ be a metric space and $S \subset X$.
  Show directly that if $S$ is sequentially compact
  then $S$ is limit-point compact without using the
  theorem we proved in class.
\end{problem}
\begin{answer}
  \begin{definition}
    A metric space $X$ is sequentially compact if every sequence in $X$
    has a convergent subsequence converging to a point in $X$.
  \end{definition}

  \begin{definition}
    A subset $S$ of a metric space $X$ is limit-point compact if every infinite subset of $S$
    has a limit point in $S$.
  \end{definition}

  \begin{claim}~\label{claim:5.1}
    If a subsequence of a sequence converges to a point in $S$,
    then the sequence also converges to that point.

    \begin{proof}
      Let $X$ be a metric space and $S \subseteq X$.
      Let $P \colonequals \braces{p_n}_1^\infty \subseteq S$ be a sequence in $S$,
      and let $Q \colonequals \braces{p_{n_k}}_1^\infty \subseteq P$ be a subsequence of $P$.
      Suppose $Q$ converges to $q \in S$.
      Then, for every $\epsilon > 0$, there exists $N \in \N$ such that
      $d(q_n, q) < \epsilon$ for all $n \geq N$.
      Since $Q \subseteq P$, this implies that $d(p_k, q) < \epsilon$ for
      infinitely many distinct $k \in \N$,
      hence $d(p_k, q) > \epsilon$ for finite $k \in \N$.
      Pick $K = \max \set{k \in \N \given d(p_k, q) > \epsilon}$.
      Then, for all $n \geq K$, $d(p_n, q) < \epsilon$,
      so $P$ also converges to $q$.
    \end{proof}
  \end{claim}

  \begin{claim}
    If $S$ is sequentially compact then $S$ is limit-point compact.

    \begin{proof}
      For any infinite subset of $S$,
      we can construct a sequence $P$ in $S$ by picking any arbitrary
      element $p_1$ in the subset,
      then picking any arbitrary element $p_2$ in the subset that is not $p_1$,
      and so on.
      Since the subset is infinite, we can always pick an element that is not
      in the sequence so far.

      \step
      Suppose $S$ is sequentially compact.
      Then, by definition of sequential compactness,
      every such sequence $P$ in $S$ has a convergent subsequence
      $Q$ converging to some point $q \in S$.
      By claim~\ref{claim:5.1}, $P$ also converges to $q$,
      so $P$ itself has a limit point in $S$. \\
      Therefore, $S$ is limit-point compact.
    \end{proof}
  \end{claim}
\end{answer}
