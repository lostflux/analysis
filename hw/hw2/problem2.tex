\begin{problem}
  Let $F = \set{0, 1, 2}$.
  Prove that there is exactly one way to define
  addition and multiplication so that $F$ is a field
  if $0$ is the additive identity and $1$ is the multiplicative identity.
\end{problem}

\begin{answer}

  \textbf{Multiplication} \\
  First, note that $0 \cdot a = a \cdot 0 = 0$ for all $a \in F$. \\
  This is because multiplication distributes over addition,
  so $0 \cdot a = (1 + (-1)) \cdot a = a + (-a) = 0$.

  \step
  Next, since $1$ is the multiplicative identity, $1 \cdot a = a \cdot 1 = a$ for all $a \in F$.

  \step
  There are three possible ways to define $2 \cdot 2$:

  \begin{enumarabic}
    \item \crim{If $2 \cdot 2 = 0$, then multiplying both sides by $2^{-1}$ implies that $2 = 0$.
      This is a contradiction.}
    \item \crim{If $2 \cdot 2 = 2$, then multiplying both sides by $2^{-1}$ implies that $2 = 1$.
      This is a contradiction.}
    \item Therefore, multiplication is only well-defined if $2 \cdot 2 = 1$.~\label{step:3times}
  \end{enumarabic}

  \step
  Since $0$ is the additive identity, $0 + a = a + 0 = a$ for all $a \in F$. \\
  Similarly, $ 1 \cdot a = a \cdot 1 = a$ for all $a \in F$.

  \step
  \textbf{Addition} \\
  There are three potential ways to define addition:
  \begin{enumarabic}
    \item \crim{If $1 + 1 = 0$, this implies $1 = -1$. \\
      For addition to be well-defined, we must also have $2+2 = 0$
      since $2$ must also have an additive inverse. \\
      How do we define $1 + 2$?
      \begin{enumroman}
        \item If $1 + 2 = 0$, then $1 = -2$, suggesting that $1$ has two additive inverses
          since $1 + 1 = 1 + 2 = 0$. But each element must have a unique inverse, so this is a contradiction.
        \item If $1 + 2 = 1$, then adding $-1$ to both sides implies that $2 = 0$, a contradiction.
        \item If $1 + 2 = 2$, then adding $-2$ to both sides implies that $1 = 0$, a contradiction.
      \end{enumroman}}
    \item If $1 + 1 = 2$. then $1 + 2 = 2 + 1 = 0$ (since both $1$ and $2$ must have additive inverses). \\
      This is not consistent with multiplication as defined in~\ref{step:3times} above,
      since:
      \begin{enumroman}
        \item $2 \cdot (1 + 2) = 2 \cdot 0 = 0$.
        \item $2 \cdot (1 + 2) = 2 \cdot 1 + 2 \cdot 2 = 2 + 1 = 0$
      \end{enumroman}
    \item \crim{$1 + 1 = 1$ implies that $1 = 0$ (by adding $-1$ to both sides).
      This is a contradiction, since $1 \neq 0$.}
  \end{enumarabic}

  Thus, there is only one way to define addition and multiplication such that $F$ is a field:

  \begin{align*}
    0 + 0 &= 0 \\
    0 + 1 = 1 + 0 &= 1 \\
    0 + 2 = 2 + 0 &= 2 \\
    1 + 1 &= 2 \\
    1 + 2 = 2 + 1 &= 0 \\
    2 + 2 &= 1 \\ \\
    0 \cdot 0 &= 0 \\
    0 \cdot 1 = 1 \cdot 0 &= 0 \\
    0 \cdot 2 = 2 \cdot 0 &= 0 \\
    1 \cdot 1 &= 1 \\
    1 \cdot 2 = 2 \cdot 1 &= 2 \\
    2 \cdot 2 &= 1
  \end{align*}



\end{answer}
